\documentclass[11pt,twoside,a4paper]{article}
\begin{document}
\section{Mini Lecture B10 - Equivalence of regular languages}
A problem is effectively soluble if there is an algorithm that always provides the answer in a finite number of steps, no matter what the inputs are. The maximum number of steps must be predictable before commencing execution procedure.
\\
\\
An effective solution to a problem that has a yes or no answer is called a decision procedure. A problem that has a decision procedure is called a decidable.
\subsection{Do two FSAs accept the same language?}
We can test whether the languages are equivalent by : \\
\begin{enumerate}
	\item Creating the difference automaton, which accepts strings that are in either one of the languages but not in the other
	\item Testing whether this automaton accepts any strings
\end{enumerate}
If the difference automaton accepts no strings, then the languages are equivalent.\\
If we can find effective procedures for 1 and 2, then testing whether two automata accept the same language is decidable.
\subsection{Transforming the equivalence problem}
\begin{itemize}
	\item For any two languages $L_1$ and $L_2$ accepted by FSAs, we can use the techniques from earlier to produce an FSA that accepts \\($L_1 - L_2) \cup (L_2 - L_1$)
	\item This is the language consisting of all strings that are in $L_1$ but not in $L_2$ or in $L_2$ but not in $L_1$
	\item If $L_1$ and $L_2$ are the same language, then the automaton will not accept any strings
	\item So, in order to make this into an effective procedure, we need to show that we can test whether an automaton accepts the empty language
\end{itemize}

\section{Mini Lecture B11 - Finiteness of regular languages}
\subsection{Testing for (in)finiteness}
Theorem : Let F be an FSA with n states.\\
If F accepts infinitely many strings, then F accepts some strings w such that $n\leq \left | w \right | < 2n$
\begin{itemize}
	\item If F accepts an infinite language then F contains at least one loop.
	\item Choose a path which has just one loop.
	\item The length of this loop cannot be greater than n.
	\item The one can construct a path the length at least n but less than 2n by going round the circuit the required number of times.
\end{itemize}

\end{document}